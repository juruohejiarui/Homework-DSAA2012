\documentclass{article}

% if you need to pass options to natbib, use, e.g.:
%     \PassOptionsToPackage{numbers, compress}{natbib}
% before loading neurips_2025

% The authors should use one of these tracks.
% Before accepting by the NeurIPS conference, select one of the options below.
% 0. "default" for submission
 \usepackage[position, final]{neurips_2025}
% the "default" option is equal to the "main" option, which is used for the Main Track with double-blind reviewing.
% 1. "main" option is used for the Main Track
%  \usepackage[main]{neurips_2025}
% 2. "position" option is used for the Position Paper Track
%  \usepackage[position]{neurips_2025}
% 3. "dandb" option is used for the Datasets & Benchmarks Track
 % \usepackage[dandb]{neurips_2025}
% 4. "creativeai" option is used for the Creative AI Track
%  \usepackage[creativeai]{neurips_2025}
% 5. "sglblindworkshop" option is used for the Workshop with single-blind reviewing
 % \usepackage[sglblindworkshop]{neurips_2025}
% 6. "dblblindworkshop" option is used for the Workshop with double-blind reviewing
%  \usepackage[dblblindworkshop]{neurips_2025}

% After being accepted, the authors should add "final" behind the track to compile a camera-ready version.
% 1. Main Track
 % \usepackage[main, final]{neurips_2025}
% 2. Position Paper Track
%  \usepackage[position, final]{neurips_2025}
% 3. Datasets & Benchmarks Track
 % \usepackage[dandb, final]{neurips_2025}
% 4. Creative AI Track
%  \usepackage[creativeai, final]{neurips_2025}
% 5. Workshop with single-blind reviewing
%  \usepackage[sglblindworkshop, final]{neurips_2025}
% 6. Workshop with double-blind reviewing
%  \usepackage[dblblindworkshop, final]{neurips_2025}
% Note. For the workshop paper template, both \title{} and \workshoptitle{} are required, with the former indicating the paper title shown in the title and the latter indicating the workshop title displayed in the footnote.
% For workshops (5., 6.), the authors should add the name of the workshop, "\workshoptitle" command is used to set the workshop title.
% \workshoptitle{WORKSHOP TITLE}

% "preprint" option is used for arXiv or other preprint submissions
 % \usepackage[preprint]{neurips_2025}

% to avoid loading the natbib package, add option nonatbib:
%    \usepackage[nonatbib]{neurips_2025}

\usepackage[utf8]{inputenc} % allow utf-8 input
\usepackage[T1]{fontenc}    % use 8-bit T1 fonts
\usepackage{hyperref}       % hyperlinks
\usepackage{url}            % simple URL typesetting
\usepackage{booktabs}       % professional-quality tables
\usepackage{amsfonts}       % blackboard math symbols
\usepackage{nicefrac}       % compact symbols for 1/2, etc.
\usepackage{microtype}      % microtypography
\usepackage{xcolor}         % colors

% Note. For the workshop paper template, both \title{} and \workshoptitle{} are required, with the former indicating the paper title shown in the title and the latter indicating the workshop title displayed in the footnote. 
\title{Cantonese Lyric Generation with Tone-Aware Alignment}


% The \author macro works with any number of authors. There are two commands
% used to separate the names and addresses of multiple authors: \And and \AND.
%
% Using \And between authors leaves it to LaTeX to determine where to break the
% lines. Using \AND forces a line break at that point. So, if LaTeX puts 3 of 4
% authors names on the first line, and the last on the second line, try using
% \AND instead of \And before the third author name.


\author{%
    Jiarui HE \\
    BEng (AI) \\
    Hong Kong University of Technology and Science (Guangzhou) \\
    Guangdong, China \\
    \texttt{jhe218@connect.hkust-gz.edu.cn} \\
}


\begin{document}

\maketitle

\begin{abstract}
Generating singable Cantonese lyrics is challenging due to tonal constraints that affect meaning and rhythm. This project proposes a \textbf{tone-aware lyric evaluation} that evaluate whether the Cantonese lyrics aligns with melody tones, and maybe a \textbf{lyric rewriting system} that generates tone-aligned Cantonese lyrics from existing songs.
\end{abstract}

\section{Introduction}
Cantonese lyric adaptation faces a unique challenge --- tonal mismatches between melody and lyrics. Unlike Mandarin, Cantonese has six to nine tones that directly affect the semantic and rhythmic harmony of songs. Current AI lyric generators often ignore tone constraints, leading to unnatural pronunciation when sung. This project aims to develop a \textbf{Evaluation of Cantonese lyrics} and \underline{maybe} a \textbf{tone-aware lyric rewriting system} that converts existing lyrics into singable Cantonese versions according to given prompts while preserving rhythm. I will mainly focus on evaluation, for the rewriting part, I will try my best to achieve it after finishing the evaluation part.

\section{Research Goals}
First step is training a state-of-the-art model to evaluate whether the lyrics has tonal mismatches. The second step is to build a model capable of generating tone-aligned Cantonese lyrics from existing songs. Final step is evaluate the quality of tone alignment.

\section{Propose Methods}
\begin{itemize}
    \item[1] Collecting and labeling songs and lyrics from websites and existing datasets.
    \item[2] SL for evaluation model. Maybe I will fine-tuned pre-trained models like QWen-Audio.
    \item[3] Both SL and RL for lyric rewriting. I will use pre-trained language models to generate lyrics, and use the evaluation model as a reward model for RL.
\end{itemize}

\section{Expected Outcomes}
\begin{itemize}
    \item Tone-aware evaluation system for Cantonese lyrics.
    \item Tone--melody alignment generation model for Cantonese lyrics.
\end{itemize}

This work demonstrates how AI can support creative music generation in tonal languages, combining \textbf{academic novelty} with \textbf{practical applications} in songwriting.

\end{document}